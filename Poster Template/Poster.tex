\documentclass[final,hyperref={pdfpagelabels=false}]{beamer}
\usepackage{grffile}
\mode<presentation>{\usetheme{I6pd2}}
\usepackage[english]{babel}
\usepackage[latin1]{inputenc}
\usepackage{amsmath,amsthm, amssymb, latexsym}
%\usepackage{times}\usefonttheme{professionalfonts}  % obsolete
%\usefonttheme[onlymath]{serif}
\boldmath
%\usepackage[orientation=portrait,size=a0,scale=1.4,debug]{beamerposter}
\usepackage[orientation=landscape,size=custom,width=121.92,height=91.44,scale=1.61,debug]{beamerposter}      

% change list indention level
% \setdefaultleftmargin{3em}{}{}{}{}{}


%\usepackage{snapshot} % will write a .dep file with all dependencies, allows for easy bundling

\usepackage{array,booktabs,tabularx}
\newcolumntype{Z}{>{\centering\arraybackslash}X} % centered tabularx columns
\newcommand{\pphantom}{\textcolor{ta3aluminium}} % phantom introduces a vertical space in p formatted table columns??!!

\listfiles

%%%%%%%%%%%%%%%%%%%%%%%%%%%%%%%%%%%%%%%%%%%%%%%%%%%%%%%%%%%%%%%%%%%%%%%%%%%%%%%%%%%%%%
\graphicspath{{fig/}}
 
\title{\LARGE TITLE GOES HERE\\[.4em]~}
\author{ \Large Authors}
\institute[ND]{\Large Dept. of Aerospace and Mechanical Engineering\\[.5em] University of Notre Dame, Notre Dame, IN}
\date[May 14, 2020]{May 14, 2020}

%%%%%%%%%%%%%%%%%%%%%%%%%%%%%%%%%%%%%%%%%%%%%%%%%%%%%%%%%%%%%%%%%%%%%%%%%%%%%%%%%%%%%%
\newlength{\columnheight}
\setlength{\columnheight}{69cm}

\def \myskip {1.8ex}
\def \myskips {2.2ex}

%%%%%%%%%%%%%%%%%%%%%%%%%%%%%%%%%%%%%%%%%%%%%%%%%%%%%%%%%%%%%%%%%%%%%%%%%%%%%%%%%%%%%%
\begin{document}
	\begin{frame}
	\begin{columns}
		% ---------------------------------------------------------%
		% Set up a column 
		\begin{column}{.32\textwidth}
			\begin{beamercolorbox}[center,wd=\textwidth]{postercolumn}
				\begin{minipage}[T]{.95\textwidth}  % tweaks the width, makes a new \textwidth
					\parbox[t][\columnheight]{\textwidth}{ % must be some better way to set the the height, width and textwidth simultaneously
						% Since all columns are the same length, it is all nice and tidy.  You have to get the height empirically
						% ---------------------------------------------------------%
						% fill each column with content            
						
						\begin{block}{Col 1}
																		
							
						\end{block}
											
						\begin{block}{Col 2}
												
							
							
						\end{block}
						
						
					}
				\end{minipage}
			\end{beamercolorbox}
		\end{column}
		% ---------------------------------------------------------%
		% end the column
		
		% ---------------------------------------------------------%
		% Set up a column 
		\begin{column}{.32\textwidth}
			\begin{beamercolorbox}[center,wd=\textwidth]{postercolumn}
				\begin{minipage}[T]{.95\textwidth}  % tweaks the width, makes a new \textwidth
					\parbox[t][\columnheight]{\textwidth}{ % must be some better way to set the the height, width and textwidth simultaneously
						% Since all columns are the same length, it is all nice and tidy.  You have to get the height empirically
						% ---------------------------------------------------------%
						% fill each column with content            
						
						\begin{block}{Col 3}
							
							
						\end{block}
						
						\begin{block}{Col 4}
							
							
							
						\end{block}
						
						
					}
				\end{minipage}
			\end{beamercolorbox}
		\end{column}
		% ---------------------------------------------------------%
		% end the column
		
		% ---------------------------------------------------------%
		% Set up a column 
		\begin{column}{.32\textwidth}
			\begin{beamercolorbox}[center,wd=\textwidth]{postercolumn}
				\begin{minipage}[T]{.95\textwidth}  % tweaks the width, makes a new \textwidth
					\parbox[t][\columnheight]{\textwidth}{ % must be some better way to set the the height, width and textwidth simultaneously
						% Since all columns are the same length, it is all nice and tidy.  You have to get the height empirically
						% ---------------------------------------------------------%
						% fill each column with content            
						
						\begin{block}{Col 5}
							
							
						\end{block}
						
						\begin{block}{Col 6}
							
							
							
						\end{block}
						
						
					}
				\end{minipage}
			\end{beamercolorbox}
		\end{column}
		% ---------------------------------------------------------%
		% end the column
	\end{columns}
	\end{frame}
\end{document}